% {{{ Preamble
\documentclass[pdftex,12pt,a4papaer]{article}
\usepackage[pdftex]{graphicx}
\usepackage{fancyhdr}
\usepackage{setspace}
\usepackage{msctexen}
\usepackage{hyperref}
\usepackage{parskip}
\usepackage{amsmath}
\usepackage{titling}
\pagestyle{fancy}
% }}}

\author{Steven Allen}
\date{\today}
\title{Protocols}

% {{{ Header
\rhead{\thedate}
\chead{\thetitle}
\lhead{\theauthor}
% }}}

\begin{document}
\maketitle

\hfill

\section{Analysis Variables}


\begin{description}
    \item[R] Number of resources
    \item[G] Number of groups per person
    \item[F] Number of friends per person
    \item[S] Number of services
    \item[M] Average members per group
    \item[r] Request rate
    \item[g] Group change rate
    \item[p] Resource post rate
\end{description}

\pagebreak

\section{Obvious PK} %{{{

\subsection{Protocol}

\begin{msc}
msc {
    a [label="Auth Agent"],
    c [label="Consumer"],
    s [label="Service"];

    c -> s [label="Request"];
    c <- s [label="Ea(res, skey, r)"];
    a <- c [label="Ea(res, skey, r), cid"];
    a -> c [label="Ec(Es(Sa(res, access?, r)))"];
    c -> s [label="Es(Sa(res, access?, r))"];
    c <- s [label="resource"];
}
\end{msc}

\subsection{Costs}

\begin{tabular}{l|l|l|l}
 Op & Auth Agent & Consumer & Service \\ \hline
 Encryption & $\Theta(r)$ & $\Theta(r)$ & $\Theta(r)$ \\
    Storage & $\Theta(RG+RS)$ & $\Theta(1)$ & $\Theta(R)$ \\
    Transfer & $\Theta(r)$ & $\Theta(r)$ & $\Theta(r)$
\end{tabular}

\subsection{Analysis}

Perfect privacy.

\pagebreak
%}}}

\section{Obvious Shared Secret}  %{{{

\subsection{Protocol}

\begin{msc}
msc {
    a [label="Auth Agent"],
    c [label="Consumer"],
    s [label="Service"];

    c -> s [label="Request"];
    c <- s [label="Es(res, r)"];
    a <- c [label="Ec(Es(res, r))"];
    a -> c [label="Ec(Es(res, access?, r))"];
    c -> s [label="Es(res, access?, r)"];
    c <- s [label="resource"];
}
\end{msc}

\subsection{Costs}

\begin{tabular}{l|l|l|l}
 Op & Auth Agent & Consumer & Service \\ \hline
 Encryption & $\Theta(r)$ & $\Theta(r)$ & $\Theta(r)$ \\
    Storage & $\Theta(RG+RS)$ & $\Theta(F)$ & $\Theta(R)$ \\
    Transfer & $\Theta(r)$ & $\Theta(r)$ & $\Theta(r)$
\end{tabular}

\subsection{Analysis}

Perfect privacy.

\pagebreak
%}}}

\section{Public Key ACL (Basic)}  %{{{

\subsection{Protocol}

\begin{msc}
msc {
    a [label="Auth Agent"],
    c [label="Consumer"],
    s [label="Service"];

    c -> s [label="Request"];
    c <- s [label="ACL"];
    c -> s [label="Token"];
    c <- s [label="resource"];
}
\end{msc}

\subsection{Costs}

\begin{tabular}{l|l|l|l}
 Op & Auth Agent & Consumer & Service \\ \hline
 Encryption & $\Theta(g)$ & $\Theta(rM)$ & $0$ \\
    Storage & $\Theta(GM)$ & $\Theta(1)$ & $\Theta(RM)$ \\
    Transfer & $\Theta(g)$ & $\Theta(rM)$ & $\Theta(rM)$
\end{tabular}

\subsection{Analysis}

The ACL is of the form: $\{E_{c_1}(t), E_{c_2}(t), ...\}$. Consumers are able to
determine the size of the ACL group.

\pagebreak
%}}}

\section{Public Key ACL (Per Group)}  %{{{

\subsection{Protocol}

\begin{msc}
msc {
    a [label="Auth Agent"],
    c [label="Consumer"],
    s [label="Service"];

    c -> s [label="Request"];
    c <- s [label="ACL"];
    c -> s [label="Token"];
    c <- s [label="resource"];
}
\end{msc}

\subsection{Costs}

\begin{tabular}{l|l|l|l}
 Op & Auth Agent & Consumer & Service \\ \hline
 Encryption & $\Theta(g)$ & $\Theta(rM)$ & $0$ \\
    Storage & $\Theta(GM)$ & $\Theta(1)$ & $\Theta(GM)$ \\
   Transfer & $\Theta(g)$ & $\Theta(rM)$ & $\Theta(rM)$
\end{tabular}


\subsection{Analysis}

Both the consumers and the services are able to group content.

\pagebreak
%}}}

\section{Public Key ACL (per group, enhanced)}  %{{{

\subsection{Protocol}

\begin{msc}
msc {
    a [label="Auth Agent"],
    c [label="Consumer"],
    s [label="Service"];

    c -> s [label="Request"];
    c <- s [label="ACL*r"];
    c -> s [label="Token XOR r"];
    c <- s [label="resource"];
}
\end{msc}

\subsection{Costs}

\begin{tabular}{l|l|l|l}
 Op & Auth Agent & Consumer & Service \\ \hline
 Encryption & $\Theta(g)$ & $\Theta(rM)$ & $0$ \\
    Storage & $\Theta(GM)$ & $\Theta(1)$ & $\Theta(GM)$ \\
    Transfer & $\Theta(g)$ & $\Theta(rM)$ & $\Theta(rM)$
\end{tabular}

\subsection{Analysis}

Using Goldwasser-Micali for pk encryption, compute $r*\text{ACL} = \{E_{c_1}(t
\oplus r), E_{c_2}(t \oplus r),...\}$. This allows us to hide the group from the
consumer.

With this encryption scheme, the consumers can't group content but the server
still can.

\pagebreak
%}}}

\section{El-Gamal Variant}  %{{{

\subsection{Protocol}

\begin{msc}
msc {
    a [label="Auth Agent"],
    c [label="Consumer"],
    s [label="Service"];

    c -> s [label="Request"];
    c <- s [label="Encrypted Token"];
    a abox c [label="(Update group keys)"];
    c -> s [label="Decrypted Token"];
    c <- s [label="resource"];
}
\end{msc}

\subsection{Costs}

\begin{tabular}{l|l|l|l}
 Op & Auth Agent & Consumer & Service \\ \hline
    Encryption & $\Theta(p+g)$ & $\Theta(r)$ & $\Theta(r+p)$ \\
    Storage & $\Theta(GM)$ & $\Theta(GF)$ & $\Theta(R)$ \\
    Transfer & $\Theta(g+p)$ & $\Theta(r)$ & $\Theta(r+p)$
\end{tabular}

\subsection{Description}

This scheme uses a variant of ElGamal that allows for multiple uncorrelatable
(to be proven) public keys per private key.

A loose description of the encryption scheme follows (there are a few other
requirements but are not necessary to get the gist of it).

The consumer key consists of two parameters: $x, w$.  The service keys are
defined as $g^{ax}, g^{a-w}$ where $a$ is chosen randomly and uniquely per
resource posted. Under this scheme, encryption is defined as $E_a(m) = (g^r,
g^{ar-wr}, mg^{axr}) = (j,k,l)$ and decryption is defined as $D_a(j, k, l) =
l/(kj^w)^x$.

The token that the service sends the consumer is $E_a(m)$, and the decrypted
token is $m$.

To prevent man-in-the-middle attacks, $m$ should encode information about the
server requesting authentication (IP address, a hash of their SSL key, etc).

Also, as ElGamal (and therefore this scheme) is malleable, the message will have
to be properly padded to ensure that it isn't modified.

\section{Privacy Analysis}

In this system, the server is unable to identify users, or even which groups
guard a piece of content. However, users are able to tell if they are in the
same group by comparing private group keys. Unfortunately, I was unable to come
up with a way around this.

\pagebreak
%}}}
\end{document}
% vim: set foldmethod=marker: %

