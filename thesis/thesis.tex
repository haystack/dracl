% {{{ Preamble
\documentclass[pdftex,12pt,a4papaer]{article}
\usepackage[pdftex]{graphicx}
\usepackage{fancyhdr}
\usepackage{setspace}
\usepackage{parskip}
\usepackage[numbers]{natbib}
\usepackage{url}
\usepackage{amsmath}
\usepackage{paralist}
\usepackage[]{units}
\usepackage{tabto}
\usepackage{float}
\usepackage{comment}
\usepackage{titling}
\usepackage{hyperref}
\usepackage[disable]{todonotes}
\usepackage{minted}
\pagestyle{fancy}
% }}}

\title{DRACL 2.0 Proposal}
\author{Steven Allen}
\date{\today}

% {{{ Header
\rhead{\thedate}
\chead{\thetitle}
\lhead{\theauthor}
% }}}

\newcommand{\note}[1]{\textit{\textbf{Note:} #1}}

\begin{document}

\thispagestyle{plain}

\begin{center}
    \vspace*{\fill}
    {%
        \onehalfspacing \bfseries \Large
        DRACL (Decentralized Resource Access Control List) \\
    }

    \vspace{\fill}
    {\large
    \begin{minipage}{0.9\textwidth}
        \emph{Author:} \theauthor \hfill \emph{Advisor:} David Karger
        \\
        \begin{center}
              \thedate
        \end{center}
    \end{minipage}
    }
    \vspace*{\fill}
\end{center}

\section{Introduction} 

DRACL is a decentralized, privacy-preserving access control system that allows
services (content hosts) to control access to content they host based on access
control lists (ACLs) provided by the content's owner (usually an end user).

% TODO: Expand

\subsection{Parties}

\begin{compactdesc}
    \item[User] An end user.
    \item[Owner] A user that publishes content and wishes to control
        access to said content.
    \item[Consumer] A user that accesses content published by a content owner
        with the owner's permission.
    \item[Authentication Provider] (AP) A service for facilitating
        authentication. Every DRACL user will have an AP (just like every email
        user has an email provider). Basically, the AP can perform limited
        actions on behalf of users when they are offline.
    \item[Content Host] A party using this system to authenticate content it
        hosts. For example, Flicker, Facebook, Imgur, etc.
\end{compactdesc}

\subsection{Goals}

This system aims to be secure, robust, privacy preserving, practical, easily deployable
and user friendly.

\subsubsection{Secure}

The system should remain secure even when parts are compromised and no
compromise should be unrecoverable.

% TODO: Expand...

\subsubsection{Robust}

The system should be robust in the face of poor network connectivity and cheap,
faulty hardware. Specifically, we aim to minimize the impact of faulty network
paths and hardware needed by DRACL alone -- network paths and hardware not
directly involved in content delivery.

% TODO
\begin{comment}
    * AP out of hot path.
    * AP does very little work.
    * CH avoids network connections.
\end{comment}

\subsubsection{Privacy Preserving}

The protocol should not leak any unnecessary information about its users.

\begin{enumerate}
    \item The authentication protocol should be zero knowledge; neither the
        consumer nor the content host learn anything except that access should
        or should not be permitted.
    \item The authentication service should learn as little as possible about
        its client's social graphs.
\end{enumerate}

\subsubsection{Practical}

To be practical, we need a business model. The only component we need to fund
directly is the AP and we plan to do so using donations. Some common ways to
support internet services are:

\begin{enumerate}
    \item Charge
    \item Advertise
    \item Sell user information
    \item Monetary donations
    \item Computer time donations (peer to peer)
\end{enumerate}

People don't pay for anything they can get for free so 1 is out. Advertising is
intrusive and annoying and we want DRACL to be as non-intrusive as possible so 2
is out. Selling user information conflicts with the our privacy goals so 3 is
out. Therefore, we are left with 4 and 5: monetary and computer time donations.

Therefore, one of the focuses in this protocol will be making the AP as
efficient and decentralized as possible.

\subsubsection{Deployable}

The system must be easy to deploy on content hosts. In order to be successful,
it must be easier for developers to deploy DRACL than a custom password based
authentication and access control scheme.

\subsubsection{User Friendly}

Any security system that gets in users way is doomed from the start. Therefore,
DRACL aims to be user friendly and unobtrusive.

\section{Related Work and Motivation}

This section discusses related work, existing systems, and the motivation behind
DRACL\@. In short, DRACL provides better usability and privacy guarantees than
any existing fully specified system.

There are five categories of authentication and access control systems:

\begin{compactitem}
\item Site specific authentication
\item Centralized authentication
\item Centralized access control
\item Decentralized authentication
\item Decentralized access control
\end{compactitem}

\subsection{Site Specific Authentication}

The vast majority of websites today force users to create site-specific
accounts. This is a poor solution to the access control problem for both users
and websites because it adds unnecessary overhead, privacy concerns, and
security hazards.

Site specific accounts add unnecessary overhead because they force websites to
implement custom account/access control systems and force users to create new
accounts for every site they use. While there are many plug-and-play
authentication solutions, they still force the website to either manually create
or implement a system for creating user accounts. DRACL will avoid this by
allowing users to create a single account with a single Authentication Provider for
use on multiple websites.

The site specific account system as an access control mechanism is a privacy
hazard because it forces content publishers to reveal their social networks.
DRACL will reduce this problem by limiting a websites view of the publisher's
social network.

Finally, site specific accounts are a security hazard because websites are
notoriously bad at safely storing credentials and users are notoriously bad at
choosing/remembering safe passwords. Because of this, websites often store user
credentials in the clear~\cite{plaintext} and users often reuse passwords and/or
use weak passwords~\cite{ms-passwords}. DRACL will reduce these security hazards
by restricting credential checking to APs and reducing the number of credentials
that users have to remember to one (the one they need to authenticate to their
AP).

\subsection{Centralized Authentication}

Centralized authentication systems such as those provided by Google and Facebook
allow users to authenticate to many sites using a single account. They therefore solve
the security problems noted above and many of the usability problems however,
these systems still force users to manually replicate parts of their social
networks on content hosts. Additionally, these systems force content hosts to
implement their own access control system.  Finally, because these systems are
centralized, users are forced to choose between a few ``accepted'' providers and
can't run their their own.

\note{Facebook allows users to give content hosts access to their social
networks. This somewhat alleviates the usability problem but introduces an
additional privacy concern: the content host learns the producers entire
social network.}

\subsection{Decentralized Authentication}

Decentralized authentication systems such as OpenID~\cite{openid} allow users to
authenticate to multiple services using the same credentials but, unlike centralized
authentication systems, decentralized authentication systems do not force users
to choose between a few ``accepted'' providers. Decentralized authentication
addresses the final concern noted above but none of the usability problems.

\subsection{Centralized Access Control}

Centralized access control systems such as Kerberos~\cite{kerberos} and
LDAP~\cite{ldap} allow services to offload user/group management to third
parties (the access control provider). This solves the usability problems noted
in the centralized authentication section because the social network is now
defined in a centralized system (the access control system). However, it makes
the centralization problem much worse because, while content hosts can allow
different users to authenticate with different centralized authentication
services, they cannot allow users to choose their own centralized
access control services without partitioning the social network by access
control service because the access control services don't interoperate.

\subsection{Decentralised Access Control}

Decentralized access control systems such as the one proposed in this document
offer the same benefits as centralized access control systems but without the
drawbacks of being a centralized system. That is, users can freely choose
between providers or run their own. 

Other systems have attempted to address this problem including \cite{attrib}
\cite{privattrib} \cite{drbac} \cite{socnet} and the Kerberos Consortium's User
Managed Access (UMA)~\cite{uma}. Unfortunately, while both \cite{attrib} and
\cite{privattrib} describe cryptographic protocols for privacy preserving
decentralized access control, neither attempt to design an implementable system.
UMA\cite{uma} \cite{drbac} and \cite{socnet} provide a more thorough system
designs (UMA even provides a system specification) but all fail to even address
the question of user privacy.

\section{Design Proposal}

This section describes the core design and does not include concrete APIs. To
make this design easy to follow, we'll first map the abstract publisher and
consumer to concrete entities:

\begin{compactitem}
    \item The publisher is Penny.
    \item The consumers are Kevin and Cary.
\end{compactitem}

Let's also assume that Penny has a stalker, Eve, who wants to see everything she
(Penny) posts.

Penny has a piece of content she wishes to share with Kevin and Cary using a
third party service. For this story, let's say she wants to share a photo with
Kevin and Cary using FoShare as the third party service (the content host).
However, there's a catch: she doesn't want her stalker, Eve, to see the photo.
Therefore, along with her photo, she needs to give FoShare some way to
distinguish ``friend from foe''. We'll call this extra information an ACL (access
control list). The following sections describe how this can be done.

\subsection{``Friending''}

Before they can securely share content with each other, Penny, Kevin, Cary, and
Eve will need some way to securely and identify each other. In this system,
they'll each use an asymmetric keypair. We will use ECC (elliptic curve
cryptography) GnuPG keypairs as GnuPG is the de-facto decentralized PKI standard
and elliptic curve cryptographic schemes produce very short signatures.

Now that they all have unique, provable identities, they need to share these
identities with each other so that they can talk about each other. They do this
by exchanging public keys. This is similar to Facebook's process of friending,
however, unlike Facebook friending, this process doesn't need to be symmetric
and doesn't inherently give Penny, Kevin, Eve, or Cary access to each other's
content.  We plan to use services like keybase.io~\cite{keybase} to help with
the initial key-exchange process.

Now that they can securely identify each other, Penny the publisher could just
tell FoShare to let Kevin and Cary (the consumers) access her photo
(identifying them by their public keys). However, this would introduce a privacy
concern because flicker would be able to learn the identities of the two
consumers.

Instead, we use an anonymous secure group communication scheme as described in
\cite{acp2} (based on an Access Control Polynomial (ACP) scheme described in
\cite{acp}). ACP will allow us to efficiently encrypt a shared secret token with
multiple secret keys without revealing anything about the secret keys. To be
able to use ACP, Penny the publisher must first generate secret keys (called
consumer keys) for every friend with whom she wishes to share content and
distribute them using her AP (described in the next section).

\subsection{Authentication Provider (AP)}

The Authentication Provider (AP) is simply our way to deal with the fact that
publishers and consumers aren't always online and, even when they are, they are
often hidden behind firewalls. The AP will host consumer keys, issue
certificates of membership (as described in the Groups and Authentication 2.0
sections), and help both consumers and publishers generate ACLs, manage friend
lists, authenticate, etc. Additionally, publisher and APs will share a
``publisher secret'' symmetric key which will be useful in the following
sections. An AP is to DRACL as an Email Provider is to Email.

For now, publishers must trust their APs to not be actively curious. That is, an
actively curious AP can access any content published by a publisher. Note,
however, that publisher do not *report* published content to their APs so
passively curious APs should learn nothing about published content. We may
choose to limit AP power in the future but doing so will impact usability and
performance.

Finally, to make it possible for users to find their friends APs  given their
public keys, each user will list his or her AP in the UID section of his or her
public PGP key.

\todo{Explain?}

\subsection{Authentication 1.0}

This section describes the basic authentication protocol which we will augment
in the Authentication 2.0 section.

\subsubsection{Generating An ACL 1.0}

Now that Penny has given Kevin and Cary secret keys, she can create an ACL using
ACP\@. The ACL has three parts: a challenge, a verifier, and an encrypted
description. The challenge includes an ACP encrypted random content token, the
domain of the content host authorized to use this ACL, and the publisher's
signature. The verifier is just the content token (unencrypted). The description
is some additional data (encrypted with the publisher secret) that the publisher
and AP can use to reconstruct the ACL\@. To keep the size of ACL within
reasonable limits, we will, for now, limit the number of users that can be
granted access to a particular piece of content to 150. In total, an ACL should
be under $\unit[4]{KiB}$. In the Groups section, we will discuss methods for
sharing content with more than 150 consumers.

% NOTE: 128*150 (ACP) + 32*150 (description) + 64*8 (signature) + 128 (content
% token) + 255*8 (content host id) == 3.3KiB (which gives us room for other stuff).

\begin{figure}[H]
\begin{minted}{JavaScript}
{
    challenge: {
        data: {
            acp: ACP(content_token, [
                Kevin_Penny_Shared_key,
                Cary_Penny_Shared_Key
            ]),
            content_host: "foshare.com"
        },
        publisher_signature: ECDSA(Penny_Private_Key, data),
    },
    verifier: content_token,
    description: Encrypted(Penny_publisher_secret, [
        "Kevin",
        "Cary"
    ])
}
\end{minted}
\caption{Example ACL}
\end{figure}


\subsubsection{Authentication 1.0}

Now Kevin learns about the photo (Penny tells him through some side channel,
e.g.\ email) so he goes to FoShare and asks for it. However, FoShare needs to
verify that Kevin has permission to access the photo. To do this, FoShare simply
sends Kevin the challenge, Kevin checks Penny's signature, verifies that that
the content host listed in the challenge is FoShare to prevent MITM attacks (we
assume the communication channel is authenticated and encrypted with SSL),
decrypts the ACP, and returns the content token to FoShare. FoShare then checks
the response against its copy of the content token, and, if they match, gives
Kevin access to the content.

Now Eve tries to access the photo. From the ACL, she learns that it was
published by Penny but that's all she learns because she is unable to decrypt
the ACP\@. Furthermore, due to the privacy preserving nature of ACP, she is unable
to learn anything about the members that access to the content. In the end, she
can't access the photo because she can't recover the content token.

\note{One drawback in this scheme is that, if the content host is
compromised, all ACLs will need to be recreated. We might be able to use
asymmetric cryptography to avoid this but that would place an additional burden
on the content host (the ACLs would larger and the content host would have to
check an asymmetric signature).}

\subsubsection{Revoking Access 1.0}

To revoke access to a piece of content, Penny simply re-creates the ACL
associated with the content based on the encrypted description included in the
original ACL.

\subsection{Groups}

While the scheme defined above works, it is missing a key component found in
most access control schemes: groups. There are two primary usability benefits of
groups:

\begin{compactenum}
    \item The ability to repeatedly grant access to a common set of consumers.
    \item The ability to add/remove a consumer to/from the set of users able to
        access an entire class of content.
\end{compactenum}

Unfortunately, we were unable to efficiently achieve the second usability
benefit while preventing content hosts learning whether or not two users are in
the same "group". Therefore, we have decided to introduce two types of groups:
``cliques'' and ``organization'' (working names).

\subsubsection{Cliques}

In DRACL, a clique is a grouping of consumers known only to an individual
publisher. Cliques preserve membership privacy (it's impossible for two members
of a clique to determine whether or not they are in the same clique) however,
new clique members will not automatically be granted access to content
previously made available to the clique and removed clique members will retain
access to any content made available to the clique before the member was
removed. In short, cliques provide the first usability benefit but not the
second.

Implementation wise, cliques are simply an abstraction. When adding a clique to
an ACL, the simply adds the members of the clique to the ACL\@. This
trivially maintains the secrecy of clique membership.

\subsubsection{Organization}

An organization is analogous to a traditional POSIX group and can be used in the
same manner. Unlike clique members, organization members gain access to all
content ever made available to an organization when they become members and
lose access if their memberships are revoked.

Organizations are not simply an abstraction. Each organization has a publisher
unique permanent 64bit ID\. However, consumers never learn these IDs and content
hosts only learn about organizations used on their service. % TODO: WORDING

For example, let's say that Penny the publisher now wants to work on a her 6.824
project with Kevin and Cary. She could just give them access to each piece of
content individually but that would be cumbersome, especially if she needs to
add another classmate to the project. So, instead, she creates a ``6.824
Project'' organization and tells her AP that Kevin and Cary are members of this
organization.

Finally, we use organizations to circumvent the 150 user limit present in the
Authentication 1.0 section. Basically, we argue that if a group contains more
than 150 users, learning that two users are a member of the same group reveals
relatively little to the content host beyond the fact that they know the
publisher.

\note{We chose 150 because it is Dunbar's \cite{dunbar} number. While not a hard
limit on social group sizes, we feel this is a generous upper bound on the
number of "close friends" a user might have.}

\subsection{Authentication 2.0}

In this section, we modify the authentication scheme described in Authentication
1.0 to accommodate organizations. Organization authentication is modeled after
kerberos~\cite{kerberos}.

\subsubsection{Generating An ACL 2.0}

To accommodate organizations, we include the list of organizations (obscured)
and a shared secret for verifying membership certificates in the ACL.

First, we include an obscured list of organizations by adding a random 16 byte
nonce and a list of hashed (using MD5) \texttt{(nonce, organization\_id)} tuples
to the challenge and a list of hashed (again, using MD5)
\texttt{(content\_host\_name, organization\_ID)} to the verifier. We hash the
organization IDs in the challenge to prevent non-members from learning anything
about which organizations have been given access to the piece of content and we
hash the verifier organization IDs to prevent content hosts from colluding to
track users across services. We may additionally choose to add decoy (fake)
organizations to obscure the cardinality. Finally, we add these organization IDs
to the encrypted description section.

\note{MD5 is sufficiently secure because we don't care about collision
resistance.}

Next, we generate a shared secret using a well-known epoch published by the  the publisher must also give the content host a
shared secret. We use derive this secret from the publisher's secret
(\texttt{SHA256("content host domain"|publisher secret)}) so that content hosts
must only remember one shared secret per publisher and publishers don't have to
remember any additional shared secrets beyond the \texttt{publisher secret}.

\note{To make these ACLs ``stand alone,'' content hosts may wish to store this
shared secret along with each ACL instead of storing it once per publisher.}


The following is the example ACL from our story for the project's final report.
To make this example more general, we have included an additional organization
(``some other org'').

\begin{figure}[H]
\begin{minted}{JavaScript}
{
    challenge: {
        data: {
            nonce: some_nonce,
            acp: ACP(content_token, []), // Empty
            organizations: [
                MD5(824_project|some_nonce),
                MD5(some_other_org|some_nonce)
            ],
            content_host: "foshare.com"
        },
        publisher_signature: ECDSA(Penny_Private_Key, data),
    },
    verifier: {
        content_token: content_token,
        organizations: [
            MD5(824_project|"foshare.com"),
            MD5(some_other_org|"foshare.com")
        ] 
    },
    description: Encrypted(Penny_publisher_secret, {
        organizations: [824_project, some_other_org],
        acp: []
    })
}
\end{minted}
\caption{Example ACL 2.0}
\end{figure}

\subsubsection{Learning About Organizations}

Before being able to prove membership in an organization, consumers need to know
to which organizations they belong. To learn this, they simply authenticate with
the publisher's AP and then ask ``which organizations do I belong to?''. The AP
responds with a list of \texttt{("Organization Name", org\_id)} tuples. In our
example, Cary would receive the following from Penny's AP:
\texttt{("6.824 Project", 824\_org)}.

\subsubsection{Authentication 2.0}

Now, to access a piece of content, the consumer can either unwrap the content
token as described in Authentication 1.0 or prove membership in one of the
associated organizations. Back to the example, Cary first needs to find the
intersection between the organizations to which he belongs and the organizations
to which Penny has granted access (to the final report). To do this, Cary,
computes \texttt{MD5(nonce|$\text{org}_n$)} for each of Penny's organizations in
which he is a member. He can then find the intersection between his
organizations and the organizations listed in the ACL's challenge.

Once the consumer knows the set of intersecting organizations (assuming it's
non-empty), the browser should ask the user which, if any, organization the user
would like to use to authenticate. Assuming the consumer decides to authenticate
as a member of one of the organizations, he or she must then either obtain a
certificate of membership (for that group, for a specific content host) from the
publisher's AP or use a cached certificate. Content hosts should generally cache
knowledge of group memberships.

\note{Browsers should remember this choice and automatically authenticate with
that organization on this content host in the future.}

A certificate of membership is simply a certificate issued by the publisher's AP
and symmetrically signed with the secret shared by the content host and the
publisher (see the Generating an ACL 2.0 section) stating: ``The bearer of this
certificate is a member of \texttt{MD5(content\_host\_name, organization ID)}. issued:
\texttt{now}, expires \texttt{some time from now}.'' For example, Cary's certificate
of membership for ``foshare.com'' would be (symmetric signature omitted):

\begin{figure}[H]
\begin{minted}{JavaScript}
{
    organization: MD5(824_project | "foshare.com"),
    issued: timestamp,
    expires: timestamp + 24h,
}
\end{minted}
\caption{Example Certificate of Membership}
\end{figure}

To obtain a certificate of membership, the consumer authenticates with the
publisher's AP and then simply asks for one. It's up to the publisher to verify
that this consumer is a member of specified the organization.

Once the consumer receives the certificate, he or she presents it to the content
host. The content host checks the signature and, if it verifies and the
certificate is for an acceptable organization, the content host grants access to
the content.

\note{Again, assuming a secure channel, there is no risk of a man in the middle
attack here because this certificate is only valid at the current content host.}

Back to the story, Cary wants to access the final report but is unable to
decrypt the ACP\@. He then tries to to find an intersection between the
organizations listed in the ACL and the ones he already knows he is a member of
(consumers should cache organization memberships when possible to reduce the
load on APs). As Penny just created this organization, Cary finds no
intersection so he fetches a new list of organizations from Penny's AP and tries
again. This time, he learns that the \texttt{824\_project} organization has access
to the report so he obtains a certificate of membership in the group
\texttt{824\_project} for the content host FoShare from Penny's AP
(authenticating with his private key). He then hands this certificate over to
FoShare which checks the certificate and finally allows Cary to access the
document.

\subsubsection{Revoking Access 2.0}

Revoking access to consumers listed in the ACP is identical to the 1.0 case.
Additionally, to revoke access to an entire organization, one simply recreates
the ACL without that specific organization. However, safely removing a consumer
from an organization is a bit more difficult.

Simply removing a user from an organization is simple. If Penny wants to kick
Kevin out of her 6.824 Project group, she can simply tell her AP to issue no
more certificates of membership to Kevin. Unfortunately, this doesn't revoke
existing certificates of membership. While they will eventually expire, they
will remain valid for a period of time. So, while Penny can kick Kevin out
without any issues, she can't kick Cary out so cleanly because he currently has
a valid (``zombie'') certificate of membership (see the previous section).

To reduce the impact of ``zombie'' certificates, we introduce a freshness
constraint. Publishers record the last time they removed a member from an
organization and, when generating an ACL for a piece of content, they tell the
content host to not accept any certificates of membership issued before this
time. This way consumers removed from an organization before the publisher
gives that organization access to a piece of content will not be able to use
their ``zombie'' certificates to access that piece of content.

\note{Content hosts must be careful to include and check the issue date in
cached group memberships information as well.}

With this additional feature, Cary  will be able to continue to access old
content made available to the organization until his certificate expires but
will not be able to access new content. 

\section{Implementation Proposal}

So far, we've given a high level description of the system. This section will
give a high level overview of how we plan to implement this system and the
components we plan to build. Specifically, we plan to build:

\begin{compactitem}
\item Plugins for common web servers and frameworks. Specifically, we will
    target Apache 2.0 and python based frameworks.
\item A browser extension.
\item A JavaScript shim library for browsers without the extension.
    Functionality may be reduced and usability may be impacted on browsers
    lacking the browser extension.
\item A reference AP.
\item A reference content host (FoShare).
\end{compactitem}

\subsection{Browser $\leftrightarrow$ Content-Host API}

We plan on implementing two APIs for content hosts: an HTTP based API and a
JavaScript based API.

The HTTP based API will require browser modification (we will create an
extension to provide the necessary functionality) but will allow for the
cleanest integration with servers such as Apache. The HTTP based API will also
allow us to use this protocol outside of the browser.

We will implement the JavaScript API in both the browser extension and a
shim library. The in-browser implementation will provide the best user
experience but we will fall-back to the shim if necessary.

\subsubsection{HTTP}

For now, we do not plan to allow publishers to attach ACLs to content using the
HTTP protocol. We expect they will do so by, e.g.\ writing custom
\texttt{.htaccess} files. However, consumers will be able to authenticate over
HTTP.

When a consumer attempts to access content protected by an ACL, the content host
must return an HTTP 401 Authorization Required response and include a
\texttt{WWW-Authenticate} header with a URL where the DRACL challenge may be
found. The consumer should then download the challenge (by issuing a GET
request), generate a response (see Authentication 2.0), and then repeat the
original request, this time with an \texttt{Authorization} header containing the
response to the challenge.

Unfortunately, this scheme requires an additional GET request to fetch the
challenge because the challenge may be too large to comfortably fit in
\texttt{WWW-Authenticate} header.

\subsubsection{JavaScript}

The JavaScript API will provide the following methods:

\begin{minted}{JavaScript}
// Ask a consumer to prove that he or she is authorized to
// access a piece of content.
function DRACLAuthenticate(challenge, callback);

// Ask a publisher to create a ACL for the described content.
// The `contentDescription` argument is a short plain-text
// description of the content.
function DRACLCreateACL(contentDescription, callback);

// Ask a publisher to update an ACL. The `aclDescription`
// argument is the description from the ACL itself.
function DRACLUpdateACL(contentDescription, aclDescription, callback);
\end{minted}

\subsection{Browser $\leftrightarrow$ Authentication Provider API}

This document will not discuss the specifics of the API for communication
between consumers, publishers, and APs. While important, the design of this API
will be driven by implementation needs and is not an API that can be reasonably
designed in advance. Additionally, this API will only impact browser authors and
Authentication Providers so, while important, it doesn't have to be elegant.

\subsection{Web Server Plugin APIs}

We will target python based frameworks by building a simple library that
provides the following two methods:

\begin{minted}{python}
def create_challenge(ACL):
    """Returns the challenge field of the ACL."""
    # ...

def check_challenge(ACL, response):
    """Returns true iff the response is validates."""
    # ...
\end{minted}

We will target Apache 2.0 by providing a server plugin and integrating with
\texttt{.htaccess}.

\section{Discussion}

\subsection{Availability}

This system never requires that the publisher/consumer be online at the same
time. Both the consumer and producer will be able to use the system from behind
firewalls.

This system does require that the Authentication Provider be available most of
the time but can tolerate small windows of down-time. % TODO: Explain?

\subsection{Performance}

\subsubsection{Authentication Provider}

The communication overhead for the Authentication Provider are no worse than
Kerberos. Consumers will re-use their secret keys so they will only have to
communicate with Authentication Providers to retrieve organization membership
certificates. As this organization certificate system is modeled after Kerberos,
its performance characteristics should be similar.

Producers only need to contact their Authentication Providers when modifying their
friend-list so the \emph{overhead} should be zero. That is, managing a
friend-list always requires communication and DRACL doesn't add any additional
communication requirements.

The storage overheads per user are:

\begin{compactitem}
\item 1 Elliptic Curve Key Key Pair ($\unit[256]{bits}$)
\item 1 Publisher Secret ($\unit[256]{bits}$)
\item Two shared secrets per friend (assuming bi-directional friendship)
    ($\unit[128]{bits}$ each, estimated 300 friends per user = $\unit[9.36]{KiB}$)
    % TODO: shared secret too small?
\item Organization IDs ($\unit[64]{bits}$ each, estimated 10 per user = $\unit[70]{KiB}$).
\end{compactitem}

Conservatively, no Authentication Provider should have to store more than
$\unit[100]{KiB}$ (including miscellaneous metadata) per user.

The cryptographic overhead is minimal. The Authentication Provider must verify
an ECDSA signature to authenticate consumers once per ``session'' but can cache
a user's identity using a cookie (TBD). The AP must also compute one symmetric
signature per organization membership certificate issued but this overhead will
quickly be dwarfed by whatever crypto is used to secure the communication
channel.

\subsubsection{Content Host}

Facebook users publish 2.5 Million pieces of content per day on average in 2012
(about $\unit[500]{TB}$ of data)~\cite{fbnum}. This is about $\unit[195]{KiB}$
per piece of content so a $\unit[4]{KiB}$ overhead won't hurt too much ($2\%$
overhead).

On the other hand, tweets have an upper limit of 140 bytes so a $\unit[4]{KiB}$
overhead means that issuing per-tweet ACL's will lead to a $2826\%$ minimum
overhead. This will bump Twitter's data storage requirements up from
$\unit[65]{GiB}$ per day to $\unit[2]{TiB}$ per day (based on a 500 million
tweets per day claim made by twitter\cite{twitter}. Unfortunately, ECDSA
signatures themselves take 64 bytes so we're stuck with a $50\%$ overhead per
tweet from the signature alone (and ECDSA signatures are some of the shortest
around \todo{Cite}). Therefore, I feel that per-tweet ACLs are simply
infeasible.

Currently, all of these storage overheads translate directly into communication
overheads. We could support sending partial ACL challenges however, this just
isn't worth the trouble. Even when using ACL's large enough to support 1200
independent ACP entries (8 times the current size), batching 20 resource
requests together to reduce the effects of latency, and splitting ACL's 16 ways,
ACL splitting only yields a 6x speedup (assuming $\unit[50]{ms}$ network
latencies and the world-wide bandwidth average of
$\unit[3.6]{Mbps}$~\cite{akamai}). Once the bandwidth is bumped up to the FCC's
broadband definition ($\unit[25]{Mbps}$)~\cite{fcc} the speedup drops to 1.5x.

Again, the cryptographic overhead is minimal. The content host may have to
verify a single symmetric signature per piece of content accessed (the
pathological worst case scenario where each piece of content has been published
to a different organization). However, this overhead should again be dwarfed by
the overhead of the crypto used to secure the communication channel.

\subsection{Security Concerns}

\subsubsection{Access Delegation}

The primary security issue in this system is access delegation. That is,
consumers can delegate their access to content either intentionally or
unintentionally (due to a security breach).

A consumer can indefinitely delegate (give away) access to all content shared
with him or her by a single friend by giving away his or her shared secret. We
hope that giving away access to everything will be enough to prevent them from
doing so willingly.

A consumer can temporarily give another user access to resources published to an
organization by giving away his or her membership certificate. Fortunately, this
fine-grained delegation is only possible for short periods of time because these
certificates expire.

The biggest concern is what happens if an attacker steals a user's credentials.
Currently, there is nothing we can do to recover.

\subsection{Privacy}

The content host can map users to host-specific organization identifiers and
colluding users can determine if they are in the same organization. However, for
small groups (Cliques), group membership is completely hidden.

\subsection{Usability}

This protocol should be very easy for content hosts to implement. From the
content-host's perspective, the authentication mechanism is stateless.
Furthermore, the content-host never needs to contact the Authentication Agent
when authenticating which means that the server-side DRACL code can be
sandboxed.

% TODO

%* Handling Security Compromises
%
%  - Relying Agent (replacing compromised ACLs)
%
%  - Consumers
%
%  - Publishers
%
%* Remove users from groups.
%
%* Prevent group members from permanently delegating privileges.
%
%* Safely using credentials on untrusted hardware.
%
%* Syncing and Backing up credentials between machines.
%
%* Avoid creating one-off groups (multiple groups in an ACL).
%
%* A distributed key distribution system.
%
%* Make it possible to decrypt an {encrypted token} without trying every group
%  known key.
%
%* Prevent MITM. Currently, Kevin must trust that FoShare isn't actually Eve in
%  disguise.
%
%
%# Key distribution service.

\bibliographystyle{acm}
\bibliography{thesis}

\end{document}
% vim: set foldmethod=marker: %
